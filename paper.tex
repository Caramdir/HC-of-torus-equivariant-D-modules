\pdfoutput=1
\documentclass[english,bw]{ck-article}

\title{Hochschild cohomology of torus-equivariant D-modules}
\author{Clemens Koppensteiner}
\date{June~24, 2015}

\addbibresource{math.bib}

\usepackage{xparse}

% categories
\newcommand\cat{\mathbf}
\newcommand\catIndCoh[2][]{\cat{IndCoh}_{#1}(#2)}
\newcommand\catVect[1][]{\cat{Vect}_{#1}}
\NewDocumentCommand \catModules {O{} O{} m} {       % modules
    \cat{Mod}^{#1}_{#2}(#3)
}
\newcommand\catDMod[2][]{\cat{DMod}_{#1}(#2)}
\newcommand\catDModHol[1]{\catDMod[\mathrm{hol}]{#1}}
\newcommand\catDModMon[2]{\catDMod{#1}^{#2-\mathrm{mon}}}
\newcommand\catDModHolMon[2]{\catDModHol{#1}^{#2-\mathrm{mon}}}

\DeclareMathOperator\Hom{Hom}
\newcommand\id[1][]{{\mathrm{Id}_{#1}}}
\newcommand\isoto\similarrightarrow
\newcommand\cocolon{%                   % colon on the right side (i.e. for an adjunction)
    \nobreak \mskip 6muplus1mu \mathpunct {}\nonscript \mkern -\thinmuskip{:}\mskip 2mu\relax%
}

% spaces
\let\stack\mathbf
\DeclareMathOperator\Spec{Spec}
\newcommand\LocSys{\operatorname{LocSys}}
\newcommand\Bun{\operatorname{Bun}}
\newcommand\ps[2][]{\mathbb P^{#2}_{#1}}
\newcommand\as[2][]{\mathbb A^{#2}_{#1}}
\newcommand\Ga{\mathbb{G}_{\mathrm{a}}}
\newcommand\Gm{\mathbb{G}_{\mathrm{m}}}
\newcommand\pt{\mathrm{pt}}
\newcommand\B{\stack{B}}

% misc
\let\sheaf\mathcal
\let\shortbar\bar
\let\bar\overline
\newcommand\dual{\vee}
\DeclareMathOperator\supp{supp}
\newcommand\cx\bullet
\newcommand\proj[1]{\operatorname{pr}_{#1}}
\newcommand\rquot[2]{%
    \mathchoice%
        {\left.#1\kern-0.2ex\middle/\kern-0.3ex\lower0.7ex\hbox{$\displaystyle #2$}\right.}%
        {\left.#1\middle/#2\right.}%
        {\left.#1\middle/#2\right.}%
        {\left.#1\middle/#2\right.}%
}
\newcommand\opalg[1]{#1^{\mathrm{op}}}

% specific

\newcommand\dR{\mathrm{dR}}
\newcommand\ΓdR{Γ_{\mkern-4mu\dR}}

\newcommand{\HCoh}{\operatorname{HH}^\cx}

\newcommand\Γsub[1]{\Gamma_{\mkern-3mu#1}}
\newcommand\barΓsub[1]{\bar{\Gamma}_{\mkern-3mu#1}}

\newcommand\ls[1]{\mathcal{L} #1}
\newcommand\lsY[1]{\mathcal{L}_{\stack Y} #1}
\newcommand\cls[1]{\overline{\mathcal{L}} #1}
\newcommand\clsY[1]{\overline{\mathcal{L}}_{\stack Y} #1}
\newcommand\lsc[1]{\mathcal{L}^c #1}
\newcommand\lscY[1]{\mathcal{L}_{\stack Y}^c #1}

\newcommand\schemels[2][]{L^{#1}#2}
\newcommand\schemelsY[2][]{\schemels[#1]{_{\stack Y}#2}}
\newcommand\schemecls[2][]{\overline{L}^{#1}#2}
\newcommand\schemeclsY[2][]{\schemecls[#1]{_{\stack Y}#2}}
\newcommand\schemelsc[2][]{L^{\if\relax\detokenize{#1}\relax\else#1,\fi c} #2}
\newcommand\schemelscY[2][]{\schemelsc[#1]{_{\stack Y}#2}}

\newcommand\schemei{i'}
\newcommand\schemej{j'}
\newcommand\schemeq{q'}
\newcommand\schemebarq{\bar{q}'}
\newcommand\dropprimes{%
    \renewcommand\schemei{i}%
    \renewcommand\schemej{j}%
    \renewcommand\schemeq{q}%
    \renewcommand\schemebarq{\bar{q}}%
}
\newcommand\schemeh{h'}

\newcommand\Stab{\operatorname{Stab}}

\newcommand\goodstack{good}
\newcommand\Goodstack{Good}
\newcommand\goodness{goodness}

\newcommand\dualize{\mathbb D}

% Formatting of upper case roman numerals (in references).
\newcommand{\RomanNum}[1]{#1}

\begin{document}

\maketitle

\begin{abstract}
    We discuss how to compute the Hochschild cohomology of the category of D-modules on a quotient stack via a relative compactification of the diagonal morphism.
    We then apply this construction to the case of torus-equivariant D-modules and describe the Hochschild cohomology as the cohomology of a D-module on the loop space of the quotient stack.
    This work is motivated by a desire to understand the support theory (in the sense of \cite{BensonIyengarKrause:2008:LocalCohomologyAndSupportForTriangulatedCategories}) of D-modules on stacks.
\end{abstract}

\section{Introduction}

Given a manifold $X$ and a category of sheaves on $X$, microlocal geometry asks whether the sheaves can be localized not just on $X$ but also with respect to codirections, i.e.~on the cotangent space $T^*X$.
For example, for constructible sheaves this leads to the notion of microsupport discussed in detail in \cite{KashiwaraSchapira:1994:SheavesOnManifolds}.
More generally, given a category of sheaves on a space $X$, we can ask whether it is possible to localize them on some space that is strictly larger than $X$ itself.

Even more generally one can ask the following question: Given a $k$-linear category $\cat C$, can one find a space over which $\cat C$ localizes?
For co-complete compactly generated triangulated categories one answer is provided by \cite{BensonIyengarKrause:2008:LocalCohomologyAndSupportForTriangulatedCategories}:
To each map from a graded-commutative ring $R$ to the center of $\cat C$ the authors associate the \emph{triangulated support} functor $\supp_R$, assigning to each object $A ∈ \cat C$ a subset $\supp_R A \subseteq \Spec R$.
This construction can be used to unify various theories of support in different areas of mathematics (though it does not yield the microlocal support of constructible sheaves).

We are led to consider the universal algebra acting on $\cat C$ by this construction, i.e.~the \emph{Hochschild cohomology} of $\cat C$.
For a complete (pre-triangulated) dg category $\cat C$ the Hochschild cohomology is the dg algebra of derived endomorphisms of the identity functor of $\cat C$:
\[
    \HCoh(\cat C)
    = \operatorname{\mathbf{R}Hom}(\id[\cat C], \id[\cat C])
    = \Hom_{\cat{Funct}(\cat C, \cat C)}(\id[\cat C], \id[\cat C]).
\]
The ring $R = \bigoplus \operatorname{HH}^{2n}(\cat C)$ is commutative and hence one can define for each $A ∈ \cat C$ the support $\supp_R A$ as a subset of $\Spec R$.
Thus understanding the Hochschild cohomology of a dg category can be an important step to understanding the category itself.

This construction, applied to the category of (ind-)coherent sheaves on a (quasismooth, dg-) scheme, yields the singular support of coherent sheaves introduced in \cite{ArinkinGaitsgory:2015:SingularSupport}.
More concretely, the authors used this notion of singular support for the category $\catIndCoh{\LocSys_G}$ in their formulation of the geometric Langlands conjecture.
By Langlands duality, one should then have a matching support theory for the category $\catDMod{\Bun_G}$ and the question arises whether it is possible to formulate this theory in a way that is intrinsic to D-modules.

A first step to this -- and also a problem of independent interest -- is to understand the Hochschild cohomology of the category $\catDMod{\stack X}$ of D-modules on a stack $\stack X$.
We will review the general setup and basic properties of D-modules on (QCA) stacks in Section~\ref{sec:pre}.
The upshot is that we have an isomorphism of dg algebras
\begin{equation}
    \label{eq:intro:d-mod-hcoh}
    \HCoh(\catDMod{\stack X}) \cong \Hom_{\catDMod{X×X}}(Δ_*ω_{\stack X}, Δ_*ω_{\stack X}),
\end{equation}
where $Δ\colon \stack X × \stack X → \stack X$ is the diagonal morphism and $ω_{\stack X}$ is the dualizing module.
In particular if $\stack X$ is a (separated) scheme, then $Δ$ is a closed embedding and $(Δ^*,Δ_*)$ adjunction combined with Kashiwara's Lemma show that $\HCoh(\catDMod{\stack X})$ is isomorphic to the de Rham cohomology of $\stack X$.
However, if $\stack X$ is not an algebraic space (and hence $Δ$ is not proper) then the situation becomes more complicated.

By Verdier duality and adjunction we can always rewrite \eqref{eq:intro:d-mod-hcoh} as
\[
    \HCoh(\catDMod{\stack X}) \cong
    \opalg{\Hom_{\catDMod{X}}(k_{\stack X}, Δ^!Δ_!k_{\stack X})} =
    \opalg{\ΓdR(\stack X, Δ^!Δ_!k_{\stack X})}.
\]
It is now tempting to look at the Cartesian square
\[
    \begin{tikzcd}
        \ls \stack X \arrow{r}{p₁} \arrow{d}{p₂} & \stack X \arrow{d}{Δ} \\
        \stack X \arrow{r}{Δ} & \stack X × \stack X
    \end{tikzcd}
\]
where
\[
    \ls \stack X = \stack X ×_{\stack X × \stack X} \stack X
\]
is the (derived) loop space of $\stack X$ and try to express the Hochschild cohomology as the cohomology of some sheaf on $\ls \stack X$.
We could expect the existence of an isomorphism
\begin{equation}
    \label{eq:intro:naive}
    \ΓdR(\stack X, Δ^!Δ_!k_{\stack X}) \cong
    \ΓdR(\stack X, p_{2,!}p₁^! k_{\stack X}).
\end{equation}
We refer to this as the \enquote{naive expectation}.
Unfortunately, the two sides are in general not isomorphic (the stack $\stack X = \ps1/\Ga$ is an easy counter-example).

In Section~\ref{sec:base-change} we will investigate how to quantify the cone of the morphism
\[
    p_{2,!}p₁^! k_{\stack X} → Δ^!Δ_! k_{\stack X}
\]
and thus the failure of the naive isomorphism \eqref{eq:intro:naive} to hold.
As an application, we will prove the following theorem, giving a class of stacks where \eqref{eq:intro:naive} is indeed an isomorphism.

\begin{Thm}\label{thm:main}
    Let $G \cong \Gm^n$ be a torus acting locally linearly on a scheme $X$ of finite type over $k$.
    Then there is a canonical isomorphism of algebras
    \[
        \HCoh\bigl(\catDMod{X/G}\bigr)
        \cong
        \opalg{\ΓdR\bigl(X/G,\,p_{2,!} p₁^! k_{X/G}\bigr)},
    \]
    where the algebra structure on $\ΓdR\bigl(X/G,\,p_{2,!} p₁^! k_{X/G}\bigr)$ is induced by the groupoid structure on $\ls(X/G)$.
\end{Thm}

\subsection*{Acknowledgments}
I would like to thank my advisor, David Nadler, for the original motivation for this project and many discussions concerning it.

\section{Prerequisites}%
\label{sec:pre}

We fix an algebraically closed base field $k$ of characteristic $0$.
All stacks will be assumed to be algebraic QCA stacks over $k$.
As we will summarize below, the QCA condition ensures that the category of D-modules on stacks is well-behaved.
Thus by definition for any stack $\stack X$ we have:
\begin{itemize}
    \item The diagonal morphism $Δ\colon \stack X → \stack X × \stack X$ is schematic.
    \item There exists a scheme $Z$ with a smooth and surjective map $Z → \stack X$.
    \item $\stack X$ is quasi-compact.
    \item The automorphism groups of the geometric points of $\stack X$ are affine.
    \item The loop space (or inertia stack) $\ls \stack X = \stack X ×_{\stack X × \stack X} \stack X$ is of finite presentation over $\stack X$.
\end{itemize}
The first two conditions ensure that the stack is algebraic, the other three that it is quasi-compact with affine automorphism group (QCA).
For details on QCA stacks we refer to~\cite{DrinfeldGaitsgory:2013:FinitenessQuestions}.
Every quotient of a scheme of finite type over $k$ by an affine algebraic group is a QCA stack, and we will be mainly interested in these.

In order to correctly define categories of D-modules on stacks it is necessary to work with dg-categories.
We refer to \cite{Keller:2006:OnDGCategories} for an introduction to dg categories.
It is often convenient to regard (pretriangulated) dg categories as $k$-linear stable $(∞,1)$-categories \cite{Lurie:2009:HigherToposTheory,Lurie:2014-draft:HigherAlgebra}, which can be done via the nerve construction \cite{Cohn:arXiv:DGCategoriesAreStableInfinityCategories,Faonte:arXiv:SimplicialNerveOfAnAinfinityCategory}. %\cite[Section~1.3.1]{Lurie:2014-draft:HigherAlgebra}
We will switch between those two languages without explicitly mentioning the intervening constructions and apply results from \cite{Lurie:2014-draft:HigherAlgebra} to dg categories.
Fortunately, a superficial knowledge of dg/$∞$-categories should be sufficient for reading this article.

\subsection{D-modules on stacks}
\label{sec:pre:d-mods}

We will be primarily concerned with D-modules on (quotient) stacks.
Unfortunately there is currently no comprehensive text available that covers all the basic constructions and properties of D-modules on stacks (or even the \emph{dg category} of D-modules on schemes).
Thus we collect all the relevant properties (without proof) in this section.
The upshot is that the familiar \enquote{six functors formalism} essentially works for holonomic D-modules and schematic morphisms of stacks.

The category of D-modules on a stack $\stack X$ can be either constructed via descent \cite{BeilinsonDrifeld:unpublished:Hitchin,DrinfeldGaitsgory:2013:FinitenessQuestions} or equivalently as ind-coherent sheaves on the de Rham space of $\stack X$ \cite{GaitsgoryRozenblyum:2014:CrystalsAndDModules}.
While the first construction is more \enquote{hands on}, the second construction is often more useful from a theoretical point of view.
It is explained in detail in the upcoming book \cite{GaitsgoryRozenblyum:prelim:StudyInDAG} (see also \cite{FrancisGaitsgory:2012:ChiralKoszulDuality} for an overview).
Many basic properties of the category $\catDMod{\stack X}$ are explored in \cite{DrinfeldGaitsgory:2013:FinitenessQuestions} and most of the following assertions are taken from there.

For any morphism $f\colon \stack X → \stack Y$ the constructions yield a continuous functor $f^!\colon \catDMod{\stack Y} → \catDMod{\stack X}$ and (after some work) a not necessarily continuous functor $f_*\colon \catDMod{\stack X} → \catDMod{\stack Y}$.
If $p\colon \stack X → \pt$ is the structure map then we set
\[
    \ΓdR(\stack X,\, {-}) = p_*({-}) \colon \catDMod{\stack X} → \catVect.
\]
The functor $\ΓdR$ is representable by a D-module $k_{\stack X}$, i.e.
\[
    \ΓdR(\stack X,\, {-}) = \Hom_{\catDMod{\stack X}}(k_{\stack X},\, {-}).
\]
Again we note that $\ΓdR(\stack X,\, {-})$ is usually not continuous and hence the object $k_{\stack X}$ not compact.

Let $Δ\colon \stack X → \stack X × \stack X$ be the diagonal.
The category $\catDMod{\stack X}$ has a monoidal structure given by the tensor product
\[
    \sheaf F \otimes \sheaf G = Δ^!\bigl( \sheaf F \boxtimes \sheaf G \bigr).
\]
The unit for this monoidal structure is $ω_{\stack X} = p^! k$.

We will be mainly concerned with the subcategory of holonomic D-modules since they enjoy extended functoriality.
\begin{Def}
    A $D$-module $\sheaf F ∈ \catDMod{\stack X}$ is called \emph{holonomic} if $f^!\sheaf F$ is holonomic for any smooth morphism $f\colon Z → \stack X$ from a scheme $Z$.
    The full subcategory of holonomic D-modules will be denoted $\catDModHol{\stack X}$.
\end{Def}

The following assertions mostly follow from their corresponding counterparts for schemes.
We refer to \cite{Braverman:LecturesOnAlgebraicDmodules} for proofs in the case of non-smooth schemes.

\begin{Prop}
    Let $f\colon \stack X → \stack Y$ be a schematic morphism.
    Then $f^!$ and $f_*$ restrict to functors
    \[
        f^!\colon \catDModHol{\stack Y} → \catDModHol{\stack X}
        \quad\text{and}\quad
        f_*\colon \catDModHol{\stack X} → \catDModHol{\stack Y}.
    \]
\end{Prop}

The Verdier duality functor on schemes induces an involutive anti auto-equivalence
\[
    \dualize_{\stack X}\colon \catDModHol{\stack X}^\mathrm{op} → \catDModHol{\stack X}
\]
such that for each smooth morphism $Z → \stack X$ of relative dimension $d$ from a scheme $Z$ one has
\[
    f^! ∘ \dualize_{\stack X} \cong \dualize_{Z} ∘ f^![-2d].
\]
The Verdier duality functor then allows us to define the \emph{non-standard functors} $f_!$ and $f^*$ for any schematic morphism $f\colon \stack X → \stack Y$ by
\begin{align*}
    f^* & = \dualize_{\stack X} ∘ f^! ∘ \dualize_{\stack Y} \colon \catDModHol{\stack Y} → \catDModHol{\stack X} \\
    \intertext{and}
    f_! & = \dualize_{\stack Y} ∘ f_* ∘ \dualize_{\stack X} \colon \catDModHol{\stack X} → \catDModHol{\stack Y}.
\end{align*}
We obtain adjoint pairs $(f_!,\, f^!)$ and $(f^*,\, f_*)$.
In some situations we can identify the non-standard functors with their standard counterparts.
If $f$ is smooth of relative dimension $d$ then $f^* = f^![-2d]$.
If $f$ is proper then $f_! = f_*$ and in particular $f_*$ is left adjoint to $f^!$.
The objects $ω_{\stack X}$ and $k_{\stack X}$ are always holonomic and
\[
    \dualize_{\stack X} ω_{\stack X} = k_{\stack X}.
\]
We have $k_{\stack X} = f^* k_{\stack Y}$ and if $\stack X$ is smooth, then $k_{\stack X} = ω_{\stack X}[-2\dim \stack X]$.

We will make use of the following lemma which follows from \cite[Lemma~5.1.6]{DrinfeldGaitsgory:2013:FinitenessQuestions}.

\begin{Lem}
    For a smooth and schematic morphism $f$ the functor $f^!$ is conservative.
\end{Lem}

\begin{Prop}[{\cite[Paragraph~\RomanNum{III}.4.2.1.3]{GaitsgoryRozenblyum:prelim:StudyInDAG}}]
    \label{prop:pre:base-change}%
    Consider a Cartesian square
    \[
        \begin{tikzcd}
            \stack Z \arrow{d}{p} \arrow{r}{q} & \stack X₁ \arrow{d}{f} \\
            \stack X₂ \arrow{r}{g} & \stack Y
        \end{tikzcd}
    \]
    with schematic morphism $f$ (and hence $p$).
    Then there is a base change isomorphism
    \[
        p_* q^! \isoto g^! f_*
    \]
    of functors from $\catDMod{\stack{X₁}}$ to $\catDMod{\stack{X₂}}$.
    If furthermore $f$ (and hence $p$) is proper, then this isomorphism coincides with the natural transformation
    \[
        p_* q^! →
        p_* q^! f^! f_* =
        p_* p^! g^! f_* →
        g^! f_*
    \]
    induced by $(f_*,\,f^!)$ and $(p_*,\, p^!)$ adjunctions.
\end{Prop}

\begin{Prop}
    \label{prop:pre:projection-formula}%
    If $f\colon \stack X → \stack Y$ is a schematic morphism then the projection formula holds, i.e.~there is a functorial isomorphism
    \[
        \sheaf F \otimes f_*(\sheaf G) \cong f_*\bigl( f^! \sheaf F \otimes \sheaf G)
    \]
    for $\sheaf F ∈ \catDMod{\stack Y}$ and $\sheaf G ∈ \catDMod{\stack X}$.
\end{Prop}

\begin{Rem}
    Propositions \ref{prop:pre:base-change} and \ref{prop:pre:projection-formula} hold more generally when $f$ is merely a \enquote{safe} morphism.
    Alternatively they hold in full generality after replacing $f_*$ by the \enquote{renormalized de Rham pushforward}.
    We fill not use either notion here and refer the interested reader to \cite{DrinfeldGaitsgory:2013:FinitenessQuestions}.
\end{Rem}

For D-modules on stacks we have the usual recollement package.
Let $i\colon \stack Z \hookrightarrow \stack X$ be a closed embedding and $j\colon \stack U \hookrightarrow \stack X$ the complementary open.
We have adjoint pairs $(i_*,\, i^!)$ and $(j^!,\, j_*)$.

\begin{Prop}[{\cite[Section~2.5]{GaitsgoryRozenblyum:2014:CrystalsAndDModules}}]
    \label{prop:recollement-std}%
    There is an exact triangle of functors
    \[
        i_*i^! → \id → j_*j^!
    \]
    on $\catDMod{\stack X}$, the adjunction morphisms
    \[
        \id → i^!i_*
        \quad\text{and}\quad
        j^!j_* → \id
    \]
    are isomorphisms, the functors $j^!i_*$ and $i^!j_*$ vanish and $i_*$ and $j_*$ are full embeddings.
\end{Prop}

On holonomic D-modules we have the additional adjoint pairs $(i^*,\, i_*)$ and $(j_!,\, j^*)$.
By applying duality to Proposition~\ref{prop:recollement-std} we obtain the distinguished triangle
\[
    j_! j^* → \id → i_*i^*
\]
and the identity $i^*j_! = 0$ on holonomic D-modules.
Further, the functor $j_!$ is a full embedding $\catDModHol{\stack U} \hookrightarrow \catDModHol{\stack X}$.

It is often useful to consider the pullback of a D-module on $\stack X$ to a smooth cover.
For quotient stacks we give the resulting subcategory a special name.

\begin{Def}
    \label{def:pre:monodromic}%
    Let $X$ be a scheme with an action of an algebraic group $G$ and let $p\colon X → X/G$ be the quotient map.
    The \emph{monodromic} subcategory $\catDModMon{X}{G} ⊆ \catDMod{X}$ is the full subcategory generated by the essential image of $p^!\colon \catDMod{X/G} → \catDMod{X}$ (or equivalently by the essential image of $p^*$).
\end{Def}

For reference, we state the following theorem which will be a crucial ingredient in the proof of Theorem~\ref{thm:main}.

\begin{Thm}[Contraction principle~{\cite[Proposition~3.2.2]{DrinfeldGaitsgory:2014:OnATheoremOfBraden}}]
    \label{thm:pre:contraction_principle}%
    Let $X$ be a scheme with an action by $\Gm$ that extends to an action of the monoid $\as 1$.
    Let $i\colon X^0 \hookrightarrow X$ be the closed subscheme of $\Gm$-fixed points and let $π\colon X → X^0$ be the contraction morphism induced by the $\Gm$-equivariant morphism $\as 1 → \{0\}$.
    Then there is an isomorphism of functors
    \[
        i^* \cong π_* \colon \catDModMon{X}{\Gm} → \catDMod{X^0}.
    \]
\end{Thm}


\subsection{Monads}
\label{sec:pre:monads}

We will deduce Theorem~\ref{thm:main} from a isomorphism of monads on $\catDMod{\stack X}$.
In this section we give a short introduction to the theory of monads and the specific constructions that we will use.
However, in the interest of readability we will mainly do so informally, skipping over the intricacies of $∞$-categories.
The interested reader can find the correct $∞$-categorical formulations in the given references.

Thus we think of a \emph{monad} on a category $\cat C$ as consisting of a triple $(T, η, μ)$, where $T\colon \cat C → \cat C$ is an endofunctor of $\cat C$, and $η\colon \id_{\cat C} → T$ and $μ\colon T∘T → T$ are natural transformations such that the diagrams
\begin{equation}
    \label{eq:pre:monad-identities}
    \begin{tikzcd}
        T³ \arrow{r}{Tμ} \arrow{d}[swap]{μT} & T² \arrow{d}{μ} \\
        T² \arrow{r}[swap]{μ} & T
    \end{tikzcd}
    \quad\text{and}\quad
    \begin{tikzcd}
        T \arrow{r}{Tη} \arrow{d}[swap]{ηT} \arrow{dr}[description]{\id} & T² \arrow{d}{μ} \\
        T² \arrow{r}[swap]{μ} & T
    \end{tikzcd}
\end{equation}
commute.
Alternatively, we can think of $T$ being a monoid in the category of endofunctors of $\cat C$ with the monoidal structure given by composition of endofunctors.
This definition also gives the correct generalization to $∞$-categories \cite[Definition~4.7.0.1]{Lurie:2014-draft:HigherAlgebra}.

Let $X$ be an object of the category $\cat C$.
Then $T$ gives the vector space $\Hom_{\cat C}(X, TX)$ the structure of a dg algebra with multiplication map
\[
    (f,g) \mapsto μ_X ∘ Tf ∘ g,
\]
\[
    \begin{tikzcd}
        X \arrow{r}{g} & TX \arrow{r}{Tf} & T²X \arrow{r}{μ_X} & TX.
    \end{tikzcd}
\]
The identities~\ref{eq:pre:monad-identities} ensure that the algebra is associative and unital.

The most common source of monads is from a pair of adjoint functors $F\colon \cat C \rightleftarrows \cat D \cocolon G$.
One simply sets $T = G ∘ F$ and $η$ and $μ$ are given by the adjunction morphisms
\[
    \id[\cat C] → G ∘ F = T
    \qquad\text{and}\qquad
    T² = G ∘ (F ∘ G) ∘ F → G ∘ F = T.
\]
We note that the correct construction in more complicated in the $∞$-categorical case and refer to \cite[Section~4.7]{Lurie:2014-draft:HigherAlgebra}.
For any $X ∈ \cat C$ the algebra construction above gives an isomorphism of algebras
\[
    \Hom_{\cat C}(X, (GF)(X)) \cong
    \Hom_{\cat D}(FX, FX).
\]

Another common way to obtain monads in geometry is via a groupoid.
Recall that a groupoid $G_{\cx}$ in stacks consists of a stack $\stack G₀$ of \enquote{objects} and a stack $\stack G₁$ of \enquote{morphisms} together with
\begin{itemize}
    \item \emph{source} and \emph{target} maps $s,t\colon \stack G₁ \rightrightarrows \stack G₀$,
    \item a \emph{unit} $e\colon G₀ → G₁$,
    \item a \emph{multiplication} (or \emph{composition}) map $m\colon \stack G₁ ×_{s,\stack G₀,t} \stack G₁ → \stack G₁$,
    \item an \emph{inverse} map $ι\colon \stack G₁ → \stack G₁$,
\end{itemize}
such that
\begin{itemize}
    \item $s ∘ e = t ∘ e = \id_{\stack G₀}$,
    \item $s ∘ m = s ∘ p₂$ and $t ∘ m = t ∘ p₁$ (where $p_i\colon \stack G₁ ×_{s,\stack G₀,t} \stack G₁$ are the projection maps).
    \item $m$ is associative,
    \item $ι$ interchanges $s$ and $t$ and is an inverse for $m$,
\end{itemize}
where all identities have to be understood in the correct $∞$-categorical way \cite[Section~6.1.2]{Lurie:2009:HigherToposTheory}.

\begin{Ex}
    For our purpose the most important example is the following:
    Let $f\colon \stack X → \stack S$ be a morphism of stacks.
    We set $\stack G_0 = \stack X$ and $\stack G₁ = \stack X ×_\stack S \stack X$.
    The source and target maps are given by $p₁$ and $p₂$, the unit by the diagonal $Δ\colon \stack X → \stack X×_\stack S\stack X$, the inverse by interchanging the factors and multiplication is $p₁₃\colon \stack X ×_\stack S \stack X ×_\stack S \stack X → \stack X×_\stack S\stack X$.
\end{Ex}

Let us for the moment assume that $s$ (and hence $t$) is proper and schematic.
In this case the maps $e$ and $m$ are also proper, since $s ∘ e = \id_{\stack G₀}$ and $s ∘ m = s ∘ p₂$ are proper.
In particular the functors $e^!\colon \catDMod{\stack G₁} → \catDMod{\stack G₀}$ and $m^!\colon \catDMod{\stack G₁} → \catDMod{\stack G₁ ×_{\stack G₀} \stack G₁}$ have left adjoints given by $e_*$ and $m_*$ respectively.
This allows us to give the endofunctor $T = s_*t^!$ of $\catDMod{\stack G₀}$ the structure of a monad in the following way:

\begin{itemize}
    \item By $(e_*,e^!)$-adjunction we have a transformation
        \[
            \id = (s∘e)_*(t∘e)^! = s_*e_*e^!t^! → s_*t^! = T.
        \]
    \item Consider the following commutative diagram
        \[
            \begin{tikzcd}[column sep=small]
                {} & & \stack G₁ \arrow[bend right]{dddll}[swap]{s} \arrow[bend left]{dddrr}{t} & & \\
                & & \stack G₁ ×_{\stack G₀} \stack G₁ \arrow{u}{m} \arrow{dl}{p₂} \arrow{dr}{p₁} & & \\
                & \stack G₁ \arrow{dl}{s} \arrow{dr}{t} & & \stack G₁ \arrow{dl}{s} \arrow{dr}{t} & \\
                \stack G₀ & & \stack G₀ & & \stack G₀
            \end{tikzcd}
        \]
        with Cartesian middle square.
        Proper base change and $(m_*,m^!)$-adjunction gives a transformation
        \[
            T² =
            s_*t^!s_*t^! =
            (s∘p₂)_*(t∘p₁)^! =
            (s∘m)_*(t∘m)^! =
            s_*m_*m^!t^! →
            s_*t^! =
            T.
        \]
\end{itemize}

In the non-$∞$-categorical setting one could easily check by hand that this is indeed a monad.
To obtain the corresponding derived statement one applies an argument similar to \cite[Section~\RomanNum{II}.1.7.2]{GaitsgoryRozenblyum:prelim:StudyInDAG}.
We will discuss a version of this below.

Let now $f\colon \stack X → \stack Y$ be schematic and proper.
The Cartesian diagram
\[
    \begin{tikzcd}
        \stack X ×_{\stack Y} \stack X \arrow{r}{p_s} \arrow{d}{p_t} & \stack X \arrow{d}{f} \\
        \stack X \arrow{r}{f} & \stack Y
    \end{tikzcd}
\]
induces a groupoid with $\stack{G}₀ = \stack X$ and $\stack{G}₁ = \stack X ×_{\stack Y} \stack X$.
The above constructions now give two monads on $\catDMod{\stack{X}}$: one by $(f_*,f^!)$ adjunction and one from the groupoid structure.
The base change isomorphism
\[
    p_{t,*} p_s^! → f^! f_*
\]
gives an identification of these monads and hence of the algebras that they induce, i.e.~for any $\sheaf F ∈ \catDMod{\stack X}$ we have
\[
    \Hom(\sheaf F,\, p_{t,*} p_s^!\sheaf F) \cong
    \Hom(\sheaf F,\, f^! f_* \sheaf F) \cong
    \Hom(f_*\sheaf F,\, f_* \sheaf F).
\]

We will need to apply this construction for non-proper $f$.
Unfortunately, in this case none of the adjunctions used to define the monads are available.
We rectify this by restricting to the full subcategory of of holonomic D-modules and using the $!$-pushforward functors instead of the $*$-pushforward ones.
Of course, by doing so we do not automatically have base change isomorphisms available anymore.
Thus we have to explicitly require that all necessary base changes hold (this is usually called the Beck-Chevalley condition).

In order to formulate the condition, we need the concept of the \emph{nerve} of a groupoid.
This is the simplicial stack, also denoted $\stack{G}_\cx$, with
\[
    \stack G_i = \underbrace{\stack G₁ ×_{\stack G₀} \dotsb ×_{\stack G₀} \stack G₁}_{\text{$i$ factors}}.
\]
We refer to \cite[Section~6.1.2]{Lurie:2009:HigherToposTheory} for the correct $∞$-categorical setup.
The following lemma is now an immediate corollary of \cite[Lemma~\RomanNum{II}.1.7.1.4]{GaitsgoryRozenblyum:prelim:StudyInDAG} or \cite[Theorem~4.7.6.2]{Lurie:2014-draft:HigherAlgebra}.

\begin{Lem}
    \label{lem:pre:groupoid_monad_hol}%
    Let $f\colon \stack X → \stack Y$ be a schematic morphism of stacks and let $\stack G_\cx$ be the corresponding groupoid.
    For each map $F\colon [n] → [m]$ in $\cat{Δ}^{\mathrm{op}}$ consider the corresponding square
    \[
        \begin{tikzcd}
            \stack G_{n+1} \arrow{r}{p_s} \arrow{d}{p_{F+1}} & \stack G_n \arrow{d}{p_F} \\
            \stack G_{m+1} \arrow{r}{p_s} & \stack G_m
        \end{tikzcd}
    \]
    where the vertical arrows are induced by $F$.
    Assume that for each such square the base change composition
    \[
        p_{F+1,!} p_s^! →
        p_{F+1,!} p_s^! p_F^! p_{F,!} =
        p_{F+1,!} p_{F+1}^! p_s^!  p_{F,!} →
        p_s^! p_{F,!}
    \]
    given by the adjunction morphisms is an isomorphism of functors from the category $\catDModHol{\stack G_n}$ to $\catDModHol{\stack G_{m+1}}$.
    Then the endofunctor $p_{t,!} p_s^!$ of $\catDModHol{\stack X}$ has a canonical structure of a monad and as such is isomorphic to the adjunction monad $f^!f_!$.
\end{Lem}

\section{Hochschild cohomology}

We recall that the Hochschild cohomology of a dg category $\cat C$ is the algebra of derived endomorphisms of the identity functor,
\[
    \HCoh(\cat C) = \mathbb{R}\mathrm{Hom}(\id[\cat C],\, \id[\cat C]).
\]
For the exact definition of the category $\mathbb{R}\mathrm{Hom} = \cat{Funct}(\cat C, \cat C)$ we refer to \cite{Keller:2006:OnDGCategories}.
Instead we will give a more concrete construction via kernels which can be applied to $\catDMod{\stack X}$.
For this let us restrict to the case of co-complete dg categories and let $\cat{Funct}_{\mathrm{cont}}(\cat C, \cat C)$ be the full subcategory of $\cat{Funct}(\cat C, \cat C)$ spanned by the continuous functors.
Then, since $\id[\cat C]$ is evidently continuous, we have
\[
    \HCoh(\cat C) =
    \Hom_{\cat{Funct}_{\mathrm{cont}}(\cat C, \cat C)}(\id[\cat C], \id[\cat C]).
\]
Let us further assume that $\cat C$ is dualizable with dual $\cat C^\dual$.
Thus there is a unit map
\[
    η\colon \catVect → \cat C^\dual \otimes \cat C
\]
and a counit map
\[
    ε\colon \cat C^\dual \otimes \cat C → \catVect
\]
fulfilling the usual compatibilities (cf.~\cite[Section~2]{BenZviNadler:arXiv:NonlinearTraces}).
Let $u = η(k)$.
Then to each continuous endofunctor $F$ of $\cat C$ we can associate its kernel $\id[\cat C^\dual] \otimes F(u) ∈ \cat C^\dual \otimes \cat C$ and conversely to each kernel $Q ∈ \cat C^\dual \otimes \cat C$ we can associate the endofunctor
\[
    \cat C
    \xrightarrow{\id[\cat C] \otimes Q}
    \cat C \otimes \cat C^\dual \otimes \cat C
    \xrightarrow{ε \otimes \id[\cat C]}
    \cat C.
\]
These assignments are mutually inverse and give an equivalence of dg categories
\[
    \cat{Funct}_{\mathrm{cont}}(\cat C, \cat C)
    \cong
    \cat C^\dual \otimes \cat C.
\]
In particular, the kernel for the identity is $u$ and hence we have
\[
    \HCoh(\cat C) =
    \Hom_{\cat C^\dual \otimes \cat C}(u, u).
\]
Let us now consider the case of $\cat C = \catDMod{\stack X}$ for a stack $X$.
Let $p\colon \stack X → \pt$ be the structure morphism and $Δ\colon \stack X → \stack X × \stack X$ the diagonal.
By \cite[Section~8.4]{DrinfeldGaitsgory:2013:FinitenessQuestions} the category $\catDMod{\stack X}$ is dualizable and there is a canonical identification
\[
    \catDMod{\stack X}^\dual \otimes \catDMod{\stack X} \cong \catDMod{\stack X × \stack X}
\]
such that the unit map is given by $Δ_*p^!$, i.e.~we have
\[
    u = Δ_*ω_{\stack X}.
\]
We summarize the above discussion in the following lemma.

\begin{Lem}\label{lem:pre:hcoh}
    Let $\stack X$ be a stack.
    Then the Hochschild cohomology of $\catDMod{\stack X}$ is given by the dg algebra
    \[
        \HCoh(\catDMod{\stack X}) =
        \Hom_{\catDMod{\stack X × \stack X}}(Δ_*ω_{\stack X},\, Δ_*ω_{\stack X}).
    \]
\end{Lem}

Recall now that we assume $Δ$ to be schematic and that the dualizing module $ω_{\stack X}$ is always holonomic.
Thus we have $\dualize Δ_* ω_{\stack X} = Δ_! k_{\stack X}$.
With this we observe that
\begin{align*}
    \HCoh\bigl(\catDMod{\stack X}\bigr)
    & = \Hom_{\catDMod{\stack X × \stack X}}(Δ_* ω_{\stack X},\, Δ_* ω_{\stack X}) & &\text{(Lemma~\ref{lem:pre:hcoh})} \\
    & = \opalg{\Hom_{\catDMod{\stack X × \stack X}}(Δ_! k_{\stack X},\, Δ_! k_{\stack X})} & & \text{(duality)} \\
    & = \opalg{\Hom_{\catDMod{\stack X × \stack X}}(k_{\stack X},\, Δ^! Δ_! k_{\stack X})} & & \text{(adjunction)} \\
    & = \opalg{\ΓdR\bigl(\stack X,\, Δ^! Δ_! k_{\stack X}\bigr)},
\end{align*}
where the algebra structure on $\ΓdR\bigl(\stack X,\, Δ^! Δ_! k_{\stack X}\bigr) = \Hom_{\catDMod{\stack X × \stack X}}(k_{\stack X},\, Δ^! Δ_! k_{\stack X})$ is the one induced by the $(Δ_!,Δ^!)$-adjunction monad.
Consider the Cartesian square
\[
    \begin{tikzcd}
        \ls{\stack X} \arrow{r}{p₁} \arrow{d}{p₂} & \stack X \arrow{d}{Δ} \\
        \stack X \arrow{r}{Δ} & \stack X × \stack X
    \end{tikzcd}
\]
Let us assume for the moment that $Δ$ (and hence $p_i$) was proper.
Then $Δ_* = Δ_!$ and $p_{2,*} = p_{2,!}$ and by Section~\ref{sec:pre:monads} we have an isomorphism of monads
\begin{equation}
    \label{eq:central_iso}
    p_{2,!} p₁^! → Δ^!Δ_!,
\end{equation}
which induces an isomorphism of algebras
\[
    \ΓdR\bigl(\stack X,\, p_{2,!} p₁^! k_{\stack X}\bigr)
    →
    \ΓdR\bigl(\stack X,\, Δ^! Δ_! k_{\stack X}\bigr).
\]
Of course, if $X$ is not an algebraic space, then $Δ$ is not proper (nor is it in general smooth).
Thus in general \eqref{eq:central_iso} is not an isomorphism and there is no canonical structure of monad on $p_{2,!} p₁^!$.
We would like to apply Lemma~\ref{lem:pre:groupoid_monad_hol} to construct a monad in special cases.
Thus the goal of the next section is to give a criterion for the assumptions of Lemma~\ref{lem:pre:groupoid_monad_hol}, i.e.~for base change to hold.

\begin{Ex}
    The base change morphism \eqref{eq:central_iso} is also an isomorphism if $Δ$ is smooth.
    In particular this implies that the \enquote{naive expectation} holds for $\stack X = \B G$ for any algebraic group $G$, i.e.~we have
    \[
        \HCoh(\catDMod{\B G}) = \opalg{\ΓdR\bigl(\B G,\, p_{2,!} p₁^! k_{\B G}\bigr)}.
    \]
    An argument similar to \cite{BenZvi:mathoverflow:CohomologyOfGmodG} shows that there is a further isomorphism
    \[
        \ΓdR\bigl(\B G,\, p_{2,!} p₁^! k_{\stack X}\bigr) \cong
        \ΓdR(G, k_G)^\dual \otimes \ΓdR(\B G, k_{\B G}).
    \]
    Alternatively, we can use the identification
    \[
        \catDMod{\B G} \cong \catModules{\ΓdR(G, k_{G})^\dual},
    \]
    where the algebra structure on $\ΓdR(G, k_G)^\dual$ is induced by the group multiplication \cite[Section~7.2]{DrinfeldGaitsgory:2013:FinitenessQuestions}.
    If $G$ is reductive, then $\ΓdR(G, k_G)^\dual$ is an exterior algebra and thus its Hochschild cohomology can be computed directly.
\end{Ex}

\section{Base change}
\label{sec:base-change}
\subsection{The cone of the base-change map}\label{sec:base-change:cone}

Consider a Cartesian diagram of stacks
\[
    \begin{tikzcd}
        \stack Z \arrow{d}{p} \arrow{r}{q} & \stack X₁ \arrow{d}{f} \\
        \stack X₂ \arrow{r}{g} & \stack Y
    \end{tikzcd}
\]
with $f$ and $g$ schematic.
We have a morphism of functors $\catDModHol{\stack X₁} → \catDModHol{\stack X₂}$,
\begin{equation}
    \label{eq:base-change-morphism}
     p_! q^! → g^! f_!
\end{equation}
induced by adjunctions
\begin{equation}
    \label{eq:base-change-adjunctions}
    p_! q^! →
    p_! q^! f^! f_! =
    p_! p^! g^! f_! →
    g^! f_!.
\end{equation}
If $f$ is proper, then \eqref{eq:base-change-morphism} is an isomorphism by Proposition~\ref{prop:pre:base-change}.
To understand the behavior for non-proper $f$, we will approximate it by a proper morphism.

\begin{Def}
    A \emph{relative compactification} of a morphism $f\colon \stack X → \stack Y$ is a commutative diagram
    \[
        \begin{tikzcd}
            \stack X \arrow[hook]{r}{j} \arrow{dr}[swap]{f} & \bar{\stack X} \arrow{d}{\bar f} \\
            & \stack Y
        \end{tikzcd}
    \]
    where $j$ is an open embedding and $\bar f$ is proper.
\end{Def}

\begin{Ex}
    A famous example of such a relative compactification is Drinfeld's compactification of the morphism $\Bun_B → \Bun_G$, where $\Bun_G$ is the stack of $G$-bundles on a curve $C$ with $G$ reductive and $B$ is a Borel subgroup of $G$ \cite{BravermanGaitsgory:2002:GeometricEisensteinSeries}.
\end{Ex}

Let us assume that in the above situation there exists a relative compactification of $f\colon \stack X₁ → \stack Y$.
Let $\stack X₁^c$ be the closed complement of the open inclusion $j\colon \stack X₁ \hookrightarrow \bar{\stack X}₁$.
Similarly, we let $\bar{\stack Z} = \stack X₂ ×_{\stack Y} \bar{\stack X}₁$ and $\stack Z^c = \stack X₂ ×_{\stack Y} \stack X₁^c$.
The notation for the corresponding inclusion and projection maps is summarized in the following Cartesian diagrams.
\[
    \begin{tikzcd}
        \bar{\stack Z} \arrow{d}{\bar p} \arrow{r}{\bar q} & \bar{\stack X}₁ \arrow{d}{\bar f} \\
        \stack X₂ \arrow{r}{g} & \stack Y
    \end{tikzcd}
    \qquad\qquad
    \begin{tikzcd}
        \stack Z^c \arrow[hook]{r}{i} \arrow{d} & \bar{\stack Z} \arrow{d}{\bar q} \\
        \stack X₁^c \arrow[hook]{r} & \bar{\stack X}₁
    \end{tikzcd}
\]
We note that $\bar{\stack Z}$ is the disjoint union of the closed substack $\stack Z^c$ and the open substack $\stack Z$.

\begin{Lem}
    \label{lem:base-change-criterion}%
    The cone of the morphism~\eqref{eq:base-change-morphism} is
    \[
        \bar p_! i_*i^* \bar{q}^! j_!.
    \]
    In particular, if $i^* \bar{q}^! j_! = 0$, then~\eqref{eq:base-change-morphism} is an isomorphism of functors.
\end{Lem}

\begin{proof}
    Let $\tilde\jmath\colon \stack Z \hookrightarrow \bar{\stack Z}$ be the open inclusion complement to $i$.
    We split the adjunction in \eqref{eq:base-change-adjunctions} in two by using the compositions
    \[
        f = \bar f ∘ j
        ,\
        p = \bar p ∘ \tilde\jmath
        \text{ and }
        q = \bar q ∘ \tilde\jmath.
    \]
    Thus the adjunction $p_!q^!→ p_!q^!f^!f_!$ becomes the sequence
    \[
        p_!q^! →
        p_!q^! j^! j_! →
        p_!q^! j^! \bar f^! \bar f_! j_!.
    \]
    The equality $p_! q^! f^! f_! = p_! p^! g^! f_!$ then becomes
    \[
        p_! q^! j^! \bar f^! \bar f_! j_! =
        p_! \tilde\jmath^! \bar q^! \bar f^! \bar f_! j_! =
        p_! \tilde\jmath^! \bar p^! g^! \bar f_! j_!.
    \]
    Finally the adjunction $p_! p^! g^! f_! → g^! f_!$ becomes
    \[
        p_! \tilde\jmath^! \bar p^! g^! \bar f_! j_! =
        \bar p_! \tilde\jmath_! \tilde\jmath^! \bar p^! g^! \bar f_! j_! →
        \bar p_! \bar p^! g^! \bar f_! j_! →
        g^! \bar f_! j_! =
        g^! f_!.
    \]
    Let us apply the same adjunction morphisms in a different order.
    First the inclusions
    \[
        p_!q^!
        \xrightarrow{α}
        p_!q^! j^! j_!
        =
        \bar p_! \tilde\jmath_! \tilde\jmath^! \bar q^! j_!
        \xrightarrow{β}
        \bar p_! \bar q^! j_!,
    \]
    and then the actual base change
    \begin{equation}
        \label{eq:lem:base-change-criterion:pf:split_morphism_base_change}
        \bar p_! \bar q^! j_!
        →
        \bar p_! \bar q^! \bar f^! \bar f_! j_!
        =
        \bar p_! \bar p^! g^! \bar f_! j_!
        →
        g^! \bar f_! j_!
        =
        g^! f_!.
    \end{equation}
    We note that the adjunction $α\colon \id → j^!j_!$ is an isomorphism and the composition of the maps in \eqref{eq:lem:base-change-criterion:pf:split_morphism_base_change} is exactly the isomorphism of proper base change (cf.~Proposition~\ref{prop:pre:base-change}).
    Thus the cone of the whole composition is the same as the cone of the morphism $β$, which is given by the recollement triangle
    \[
        \bar p_! \tilde\jmath_! \tilde\jmath^! \bar q^! j_!
        \xrightarrow{β}
        \bar p_! \bar q^! j_!
        \xrightarrow{\phantom{β}}
        \bar p_! i_* i^* \bar q^! j_!
        \xrightarrow{+1}.
        \qedhere
    \]
\end{proof}

\subsection{Relative compactification for quotient stacks}
\label{sec:base-change:compactification}%

In the preceding section we simply assumed that a relative compactification of the given map exists.
We will now construct such a compactification for the diagonal map of a quotient stack.
Thus let $X$ be a scheme of finite type over $k$ and let $G$ be an affine algebraic group over $k$ acting on $X$.
Let $\stack X = X/G$ be the corresponding quotient stack.

Constructing a relative compactification of $Δ\colon \stack X → \stack X × \stack X$ is the same as first constructing a $G × G$-equivariant relative compactification of $(\proj2, a)\colon G × X → X × X$ (where $a\colon G × X → X$ is the action map) and then taking the quotient by the $G × G$ action\footnote{%
    Here $G × G$ acts on $G × X$ by $(s₁,s₂) \cdot (g,x) = (s₂gs₁^{-1},\, s₁x)$.
}.
We let
\[
    Γ = \bigl\{(g, x, x, gx) ∈ G × X × X × X\bigr\}
\]
be the graph of $(\proj2, a)$.

We pick a $G×G$-equivariant compactification $\bar G$ of $G$ and let $\bar Γ$ be the closure of $Γ$ in $\bar G × X × X × X$.
We have an open embedding $j$ of $G × X \cong Γ$ into $\bar Γ$ and proper map $f\colon \bar Γ → X × X$ given by projection on the last two factors.
The composition $f ∘ j$ is equal to $(\proj2, a)$.

Instead of viewing $Γ$ as the graph of $(\proj2, a)$ we can drop the third factor and regard $Γ$ as the graph of the action map, i.e.
\[
    \Γsub{a} = \bigl\{(g, x, gx) ∈ G × X × X\bigr\}.
\]
The closure $\barΓsub{a}$ of $\Γsub a$ in $\bar G × X × X$ identifies with $\bar Γ$.
Thus for ease of notation we will from now on always set $Γ = \Γsub a$ and $\bar Γ = \barΓsub{a}$.

\begin{Def}
    Let $\stack X = X/G$.
    With the above construction we set
    \[
        \bar{\stack X} = \rquot{\bar Γ}{G×G}.
    \]
    We have an open embedding $j\colon X \hookrightarrow \bar{\stack X}$ and a proper morphism $\shortbar Δ\colon \bar{\stack X} → \stack X × \stack X$ induced by the map $f$ above, such that $Δ = \shortbar Δ ∘ j$.
\end{Def}

\begin{Rem}
    In the case of $G = \Gm$ the compactification $\bar Γ$ is explicitly described in \cite{DrinfeldGaitsgory:2014:OnATheoremOfBraden}.
    In particular, if $X$ is smooth it is shown there that $\bar Γ$ is smooth over $\bar G = \ps1$.
    It is possible to extend the methods of \cite{DrinfeldGaitsgory:2014:OnATheoremOfBraden} to quotients by higher dimensional tori.
    The resulting constructions are useful for doing explicit computations.
\end{Rem}

It is useful to consider only partial compactifications.
For this let $V$ be a $G$-stable subvariety of $\bar G$ and let $\barΓsub{V}$ be the closure of $Γ$ in $G × V$.
We set
\[
    \bar{\stack X}_V = \rquot{\barΓsub{V}}{G×G}.
\]
Clearly, if $\left\{V_i\right\}$ is an open cover of $\bar G$ by $G$-stable subvarieties, then $\left\{\bar{\stack X}_{V_i}\right\}$ is an open cover of $\bar{\stack X}$.

\subsection{\Goodstack\ stacks}

Let $\stack X = X/G$ be a quotient stack as before.
For any morphism of stacks $h\colon \stack Y → \stack X$ we set $\lsY{\stack X} = \stack X ×_{\stack X × \stack X} \stack Y$.
Thus we have the Cartesian diagram
\[
    \begin{tikzcd}
        \lsY \stack X \arrow{r}{q_{\stack Y}} \arrow{d} & \stack X \arrow{d}{Δ} \\
        \stack Y \arrow{r}{Δ∘h} & \stack X × \stack X
    \end{tikzcd}
\]
Let us fix a relative compactification $\shortbar Δ \colon \bar{\stack X} → \stack X × \stack X$ as in Section~\ref{sec:base-change:compactification}.
Using the notation of Section~\ref{sec:base-change:cone} we set $\clsY{\stack X} = \bar{\stack X} ×_{\stack X × \stack X} \stack Y$ and $\lscY{\stack X} = {\stack X}^c ×_{\stack X × \stack X} \stack Y$.
We let $\bar q_{\stack Y}\colon \clsY\stack X → \bar{\stack X}$ be the projection morphism and $i_{\stack Y}\colon \lscY\stack X \hookrightarrow \clsY\stack X$ the inclusion.
Thus we have the following central diagram
\[
    \begin{tikzcd}
        \lsY\stack X \arrow[hook]{r} \arrow{d}{q_{\stack Y}} & \clsY\stack X \arrow{d}{\bar q_{\stack Y}} \arrow[hookleftarrow]{r}{i_{\stack Y}} & \lscY\stack X \arrow{d} \\
        \stack X \arrow[hook]{r}{j} & \bar{\stack X} \arrow[hookleftarrow]{r} & \stack X^c
    \end{tikzcd}
\]

\begin{Def}
    A quotient stack $\stack X = X/G$ is called \emph{\goodstack} if for every quotient stack $\stack Y = Y/G$ and schematic morphism $\stack Y → \stack X$ the functor $i_{\stack Y}^* \bar q_{\stack Y}^! j_!$ vanishes on $\catDModHol{\stack X}$.
\end{Def}

We will show in Section~\ref{sec:torus} that any stack of the form $\rquot{X}{\Gm^n}$ is \goodstack.
The reason for this definition is the following theorem which lets us compute the Hochschild cohomology of $\catDMod{\stack X}$ for \goodstack\ quotient stacks.

\begin{Thm}\label{thm:good-is-good}
    If $\stack X = X/G$ is \goodstack, then there exists a canonical structure of monad on $p_{2,!}p₁^!$ and the morphism $p_{2,!}p₁^! → Δ^!Δ_!$ is an isomorphism of monads.
    In particular there is an isomorphism of algebras
    \[
        \HCoh\bigl(\catDMod{\stack X}\bigr)
        \cong
        \opalg{\ΓdR\bigl(\stack X,\, p_{2,!} p₁^! k_{\stack X}\bigr)}.
    \]
\end{Thm}

In other words, Theorem~\ref{thm:main} holds for \goodstack\ stacks.

\begin{proof}
    We apply Lemma~\ref{lem:pre:groupoid_monad_hol} to the groupoid $\ls\stack X \rightrightarrows \stack X$.
    Thus we let $\stack G_\cx$ be the simplicial stack with
    \[
        \stack G_i = \underbrace{\ls\stack X ×_{\stack X} \dotsb ×_{\stack X} \ls\stack X}_{\text{$i$ factors}}.
    \]
    Any morphism $F\colon [n] → [m]$ in $\cat{Δ}^{\mathrm{op}}$ induces a diagram
    \[
        \begin{tikzcd}
            \stack G_{n+1} \arrow{d} \arrow{r} & \stack G_{m+1} \arrow{d} \arrow{r} & \stack X \arrow{d}{Δ} \\
            \stack G_{n} \arrow{r} & \stack G_m \arrow{r} & \stack X × \stack X
        \end{tikzcd}
    \]
    We have to show that base change holds along the left-hand square.
    But by the assumption and Lemma~\ref{lem:base-change-criterion}, base change holds along the outer rectangle and the right-hand square.
    Thus it also holds along the left-hand square.
\end{proof}

\begin{Rem}
    We expect that most quotient stacks are not \goodstack.
    For example, a direct computation shows that Theorem~\ref{thm:main} does not hold for the stack $\ps1/\Ga$, and hence it is not \goodstack.
    For non-\goodstack\ stacks, the sheaf $i_{\stack X}^* \bar q_{\stack X}^! j_! k_{\stack X}$ instead gives a measurement for the failure of the naive expectation for $\HCoh\bigl(\catDMod{\stack X}\bigr)$.
\end{Rem}

We finish this section with some useful observations for proving that a stack is \goodstack.

\begin{Lem}
    If $\stack X₁ = X₁/G₁$ and $\stack X₂ = X₂/G₂$ are \goodstack, then $\stack X₁ × \stack X₂$ is \goodstack.
\end{Lem}

\begin{proof}
    Follows from compatibility of $i_{\stack Y}^*\bar{q}_{\stack Y}^!j_!$ with $\boxtimes$ and coproducts.
    (Note that $\lscY{(\stack X₁ × \stack X₂)} = \lscY \stack{X₁} × \clsY \stack{X₂} ∪ \clsY{\stack X₁} × \lscY{\stack X₂}$.)
\end{proof}

\begin{Lem}
    Let $U$ be a $G$-equivariant open subset of $X$.
    If $X/G$ is \goodstack\ then $U/G$ is \goodstack.
\end{Lem}

\begin{proof}
    Let $\stack U = U/G$ and let $\stack Y$ be a quotient stack mapping into $\stack U$ (and hence also into $\stack X$).
    Consider the diagram
    \begin{equation}
        \label{eq:lem:cover:diagram}
        \begin{tikzcd}
            \lscY \stack U \arrow[hook]{r}{i_{\stack U, \stack Y}} \arrow[hook]{d}{α} & \clsY{\stack U} \arrow{r}{\bar q_{\stack U,\stack Y}} \arrow[hook]{d}{β} & \bar{\stack U} \arrow[hook]{d}{γ} \arrow[hookleftarrow]{r}{j_{\stack U}} & \stack U  \arrow[hook]{d}{δ} \\
            \lscY \stack X \arrow[hook]{r}{i_{\stack X, \stack Y}}                    & \clsY{\stack X} \arrow{r}{\bar q_{\stack X,\stack Y}}                    & \bar{\stack X}                    \arrow[hookleftarrow]{r}{j_{\stack X}} & \stack X 
        \end{tikzcd}
    \end{equation}
    The vertical arrows are open embeddings and all squares are Cartesian (where we use the same compactification of $G$ for $\bar{\stack X}$ and $\bar{\stack U}$).
    Thus
    \begin{equation*}
        i_{\stack U, \stack Y}^* \bar q_{\stack U,\stack Y}^! j_{\stack U,!} =
        i_{\stack U, \stack Y}^* \bar q_{\stack U,\stack Y}^! j_{\stack U,!} δ^* δ_*=
        α^* i_{\stack X, \stack Y}^* \bar q_{\stack X, \stack Y}^! j_{\stack X,!} δ_* =
        0.
        \qedhere
    \end{equation*}
\end{proof}

The same argument can be used to reduce the computation to a smooth cover.
We will now introduce notation for the special case of the cover $\bar Γ → \bar{\stack X}$.
The corresponding covers of the other relevant stacks are introduced in the following diagram with Cartesian squares.
\begin{equation}
    \label{eq:base-change:scheme-cover}
    \begin{tikzcd}
        \schemelscY \stack X \arrow[hook]{r}{\schemei_{\stack Y}} \arrow{d} & \schemeclsY \stack X \arrow{r}{\schemebarq_{\stack Y}} \arrow{d} & \bar Γ \arrow{d} \arrow[hookleftarrow]{r}{\schemej} & Γ \arrow{d} \\
        \lscY \stack X \arrow[hook]{r}{i_{\stack Y}}                        & \clsY \stack X \arrow{r}{\bar q_{\stack Y}}                      & \bar{\stack X}   \arrow[hookleftarrow]{r}{j} & \stack X
    \end{tikzcd}
\end{equation}
We note that all vertical morphisms are smooth and the spaces in the top row are schemes.
Let $\schemeh\colon X → Y$ be the $G$-equivariant morphism of schemes inducing $h$ on quotient stacks.
Then the scheme $\schemeclsY \stack X$ is given by
\[
    \schemeclsY \stack X =
    \biggl\{
        \bigl(g₁,\, y,\, g₂\bigr) ∈ G × Y × \bar G : \bigl(g₂,\, \schemeh(y),\, g₁\schemeh(y)\bigr) ∈ \bar Γ
    \biggr\}.
\]

\begin{Lem}\label{lem:base-change:scheme-cover}
    A stack $X/G$ is \goodstack\ if and only if for each morphism $Y/G → X/G$ the composition $(\schemei_{\stack Y})^* (\schemebarq_{\stack Y})^! \schemej_!$ vanishes on $\catDModHolMon{Γ}{G×G}$.
\end{Lem}

\begin{proof}
    Follows from the fact that pullback along the smooth vertical morphisms in \eqref{eq:base-change:scheme-cover} is conservative \cite[Lemma~5.1.6]{DrinfeldGaitsgory:2013:FinitenessQuestions} and permutes with the other morphisms up to a shift.
\end{proof}

\begin{Lem}
    \label{lem:base-change:cover}%
    If there exists a $G$-stable open cover $U_i$ of $X$ such that all stacks $U_i/G$ are \goodstack, then $X/G$ is \goodstack.
\end{Lem}

\begin{proof}
    Let $\stack U_i = U_i/G$ be the corresponding quotient stacks.
    We first show that the stacks $\lscY{\stack U_i}$ form an open cover of $\lscY{\stack X}$\footnote{
        This is not completely obvious, since the $\bar{\stack U}_i$ do not necessarily form a cover of $\bar{\stack X}$.
        For example, consider $\ps1$ with the usual linear $\Gm$-action and the usual affine cover.
    }.
    For this it suffices to show that the open subschemes $\schemeclsY \stack U_i$ cover $\schemeclsY \stack X$.
    Let $(g₁, y, g₂)$ be a point of $\schemeclsY \stack X$.
    Then there exists some $U_i$ with $h'(y) ∈ U_i$.
    But then $g₁h'(y)$ is also in $U_i$ and hence $(h'(y), g₂, g₁h'(y)) ∈ \bar{\stack U}_i$.
    Thus $(g₁, y, g₂)$ is in $\schemeclsY \stack U_i$.

    It now suffices to show that the restrictions of $i_{\stack Y}^*\bar{q}_{\stack Y}^!j_! \sheaf F$ to $\lscY{\stack U_i}$ vanish for every $\sheaf F ∈ \catDMod{\stack X}$.
    But this follows from the diagram~\eqref{eq:lem:cover:diagram} (for $\stack U_i$ instead of $\stack U$) and the \goodness\ of $\stack U_i$.
\end{proof}

Let $\left\{V_i\right\}$ be a $G$-stable open cover of $\bar G$ and consider the corresponding open cover $\left\{\bar{\stack X}_{V_i}\right\}$ of $\stack X$.
We obtain open covers $\left\{\cls_{V_i,\stack Y} \stack X\right\}$ and $\left\{\lsc_{V_i,\stack Y} \stack X\right\}$ of $\clsY{\stack X}$ and $\lscY{\stack X}$ respectively.
We let $i_{V_i,\stack X}$, $\bar q_{V_i,\stack X}$ and $j_{V_i}$ be the corresponding maps, i.e.
\[
    \begin{tikzcd}
        \cls_{V_i,\stack Y} \stack X \arrow[hook]{r}{i_{V_i,\stack X}} &
        \lsc_{V_i,\stack Y} \stack X \arrow{r}{\bar q_{V_i,\stack X}} &
        \bar{\stack X}_{V_i} \arrow[hookleftarrow]{r}{j_{V_i}} &
        \stack X.
    \end{tikzcd}
\]

\begin{Lem}\label{lem:base-change:cover-by-relative-compactifications}
    With the above notation, the $\stack X$ is \goodstack\ if and only $i_{V_i,\stack Y}^* \bar q_{V_i,\stack Y}^! j_{V_i,!}$ vanishes on $\catDModHol{\stack X}$ for all $V_i$ and all $\stack Y → \stack X$.
\end{Lem}

\begin{proof}
    Similar to the proof of Lemma~\ref{lem:base-change:cover}.
\end{proof}

\section{Torus quotients}
\label{sec:torus}

In this section we will apply the tools from the previous section to torus quotient stacks.
Specifically, we will prove the following theorem.

\begin{Thm}
    \label{thm:torus:is-good}%
    Let $G \cong \Gm^n$ be a torus acting locally linearly on a scheme $X$ of finite type over $k$.
    Then the stack $\stack X = X/G$ is good.
\end{Thm}

Together with Theorem~\ref{thm:good-is-good} this implies our main result, Theorem~\ref{thm:main}.

\begin{Rem}
    We only use the assumption that the action is locally linear to prove Lemma~\ref{lem:finitely_many_stabilizers}, i.e.~that $\Stab\stack X$ is locally finite.
    Thus it would suffice to assume that $X$ can be covered by open subschemes for which Lemma~\ref{lem:finitely_many_stabilizers} holds.
\end{Rem}

\begin{Rem}
    Theorem~\ref{thm:torus:is-good} actually holds for $G$ the product of a torus and a finite Abelian group.
    The argument is exactly the same.
\end{Rem}

By Lemma~\ref{lem:base-change:cover}, it suffices to prove Theorem~\ref{thm:torus:is-good} for stacks $X/G$ with $X$ affine.
We fix an isomorphism $G \cong \Gm^n$ and compactify $G$ to $\bigl(\ps1\bigr)^n$.
The variety $\bigl(\ps1\bigr)^n$ can be covered by $G$-equivariant open subvarieties of the form $\as n$.
Thus by Lemma~\ref{lem:base-change:cover-by-relative-compactifications}, it suffices to check \goodness\ for the partial compactification $\Gm^n \subseteq \as n$.
To simplify notation, we drop the subscript $\as n$ from the notation and set $\bar{\stack X} = \bar{\stack X}_{\as n}$ and similarly for the various maps.

We fix a quotient stack $\stack Y = Y/G$ and a morphism $h\colon \stack Y → \stack X$, induced by a $G$-equivariant morphism $\schemeh\colon Y → X$.
According to Lemma~\ref{lem:base-change:scheme-cover}, rather than working directly with the stacks, we can base change to schemes.
We will use the notation of Lemma~\ref{lem:base-change:scheme-cover}, but for ease of notation we will drop the subscript $\stack Y$ and the primes from the maps.
\dropprimes%
Thus we are concerned with the diagram
\[
    \begin{tikzcd}
        \schemelscY \stack X \arrow[hook]{r}{\schemei} &
        \schemeclsY \stack X \arrow{r}{\schemebarq} &
        \bar Γ \arrow[hookleftarrow]{r}{\schemej} &
        Γ,
    \end{tikzcd}
\]
where we have to show that $\schemei^* \schemebarq^! \schemej_!$ vanishes on $\catDModHolMon{Γ}{G×G}$.

The general idea is to introduce a $\Gm$-action that contracts $\schemelscY \stack X$ onto $\schemeclsY \stack X$.
In order to do so, we will cut the scheme
\[
    \schemeclsY \stack X =
    \biggl\{
        \bigl(g₁,\, y,\, g₂\bigr) ∈ G × Y × \bar G : \bigl(g₂,\, \schemeh(y),\, g₁\schemeh(y)\bigr) ∈ \bar Γ
    \biggr\}
\]
into pieces according to the subgroups of $G$ that stabilize $\schemeh(y)$.
For this let $\Stab\stack X$ be the set of all closed subgroups of $G$ that are stabilizers of points of $X$, i.e.
\[
    \Stab\stack X = \{ G_x : x ∈ X \}.
\]

\begin{Lem}
    \label{lem:finitely_many_stabilizers}%
    The set $\Stab\stack X$ is finite.
\end{Lem}

\begin{proof}
    Since $X$ is affine, it can be embedded $G$-equivariantly into some $\as m$ with a linear $T$-action.
    For $\as m/T$ the statement is easy to see.
\end{proof}

Let $S$ be closed subgroup of $G$ and let $X^S$ be the $S$-fixed points of $X$.
Since $G$ is Abelian (and hence $S$ a normal subgroup), $X^S$ is a $G$-stable closed subscheme of $X$.
Hence $X^S/G$ is a closed substack of $\stack X$.
Let $\bar S$ be the closure of $S$ in $\bar G = \as n$ and consider the closed subspace
\[
    \schemecls[S]_{\stack Y} \stack X =
    \biggl\{
        \bigl(g₁,\, y,\, g₂\bigr) ∈ G × Y × \bar G : \schemeh(y) ∈ X^S,\, \bigl(g₂,\, \schemeh(y),\, g₁\schemeh(y)\bigr) ∈ \bar Γ \text{ and } g₂ ∈ g₁\bar S
    \biggr\}
    ⊆
    \schemecls_{\stack Y} \stack X.
\]

\begin{Lem}
    \label{lem:stabilizers_cover}%
    The subspaces $\schemecls[S]_{\stack Y} \stack X$ for $S ∈ \Stab\stack X$ cover $\schemecls_{\stack Y}\stack X$.
\end{Lem}

\begin{proof}
    Let $\schemelsY \stack X$ be the smooth cover of $\ls_{\stack Y}\stack X$.
    Consider the spaces
    \[
        \schemelsY[S] \stack X =
        \biggl\{
            \bigl(g₁,\, y,\, g₂\bigr) ∈ G × Y × G : \schemeh(y) ∈ X^S,\, \bigl(g₂,\, \schemeh(y),\, g₁\schemeh(y)\bigr) ∈ Γ \text{ and } g₂ ∈ g₁S
        \biggr\}
        \subseteq \schemelsY \stack X.
    \]
    The closure of $\schemelsY[S] \stack X$ in $\schemeclsY \stack X$ is exactly $\schemeclsY[S] \stack X$.
    It is easy to see that the subspaces $\schemelsY[S] \stack X$ for $S ∈ \Stab \stack X$ cover $\schemelsY[S] \stack X$.
    Now the statement follows from the fact that the closure of a finite union is the union of the individual closures.
\end{proof}

It will be useful to have a slight generalization of the schemes $\schemeclsY[S] \stack X$.
Let $S₁ \subseteq S₂$ be two subgroups of $G$ contained in $\Stab\stack X$.
We set
\[
    \schemeclsY[S₁,S₂] \stack X =
    \biggl\{
        \bigl(g₁,\, y,\, g₂\bigr) ∈ G × Y × \bar G : \schemeh(y) ∈ X^{S₂},\, \bigl(g₂,\, \schemeh(y),\, g₁\schemeh(y)\bigr) ∈ \bar Γ \text{ and } g₂ ∈ g₁\bar S₁
    \biggr\}.
\]
Clearly we have $\schemeclsY[S₁,S₂]{\stack X} \subseteq \schemecls[S₁]{\stack X}$ and $\schemeclsY[S] \stack X = \schemeclsY[S,S] \stack X$.

Consider the Cartesian square of closed embeddings
\[
    \begin{tikzcd}
        \schemelscY[S₁,S₂] \stack X \arrow[hook]{r}{α_{S₁,S₂}^c} \arrow[hook]{d}{\schemei^{S₁,S₂}} & \schemelscY\stack X \arrow[hook]{d}{\schemei}
        \\
        \schemeclsY[S₁,S₂] \stack X \arrow[hook]{r}{α_{S₁,S₂}}   & \schemeclsY\stack X
    \end{tikzcd}
\]

\begin{Lem}\label{lem:key_for_stablizier}%
    For any $S₁ ⊆ S₂$ in $\Stab\stack X$ and any $\sheaf F ∈ \catDModHolMon{Γ}{G×G}$ we have
    \[
        \schemei^{S₁,S₂,*} α_{S₁,S₂}^! \schemebarq^! \schemej_! \sheaf F = 0.
    \]
\end{Lem}


\begin{proof}
    The scheme $\schemelsc[S₁,S₂] \stack X$ is given by
    \[
        \biggl\{
            \bigl(g₁,\, y,\, g₂\bigr) ∈ G × Y × \bar G : \schemeh(y) ∈ X^{S₂},\, \bigl(g₂,\, \schemeh(y),\, g₁\schemeh(y)\bigr) ∈ \bar Γ \text{ and } g₂ ∈ g₁(\bar S₁ \setminus S₁)
        \biggr\}.
    \]
    If $S₁ = \bar S₁$ the statement is trivially true.
    Otherwise the scheme $\bar S₁ \setminus S₁$ is the union of hyperplanes $H_i$ of $\bar S₁$.
    It suffices to prove the statement when further restricting to
    \[
        \biggl\{
            \bigl(g₁,\, y,\, g₂\bigr) ∈ G × Y × \bar G : \schemeh(y) ∈ X^{S₂},\, \bigl(g₂,\, \schemeh(y),\, g₁\schemeh(y)\bigr) ∈ \bar Γ \text{ and } g₂ ∈ g₁H_i
        \biggr\}.
    \]
    for all $i$.
    Let $H$ be one such hyperplane.
    We will assume that $H$ is contained in the closure of the connected component of $1 ∈ S₁$.
    The proof for $H$ in a different component is the same, up to a shift by an element of $G$.
    Let $\schemei_H$ be the inclusion of
    \[
        Z =
        \biggl\{
            \bigl(g₁,\, y,\, g₂\bigr) ∈ G × Y × \bar G : \schemeh(y) ∈ X^{S₂},\, \bigl(g₂,\, \schemeh(y),\, g₁\schemeh(y)\bigr) ∈ \bar Γ \text{ and } g₂ ∈ g₁H
        \biggr\}
    \]
    into $\schemeclsY[S₁,S₂] \stack X$.
    We want to compute
    \[
        (\schemei_H)^{*} α_{S₁,S₂}^! \schemebarq^! \schemej_! \sheaf F.
    \]
    We will do so be introducing a contractive $\Gm$-action on $\bar Γ$ and $\schemelsY[S₁,S₂]\stack X$ such that the morphism
    \[
        \schemebarq ∘ α_{S₁,S₂}\colon \schemeclsY[S₁,S₂]\stack X → \bar Γ
    \]
    is $\Gm$-equivariant.

    We write $G \cong G₁ × S₁$ for some subgroup $G₁$ of $G$.
    This gives a corresponding decomposition of the monoid $\bar G \cong \as n$ as $\bar G = \bar G₁ \bar S₁$.
    Let $H' = \bar G₁ H$.
    We note that $G ∩ H' = \emptyset$.

    We chose an action $μ$ of $\Gm$ on $\bar S₁$ that contracts $\bar S₁$ onto $H$.
    This induces an action of $\Gm$ on $\bar G = \bar G₁ \bar S₁$ by $u \cdot ts = tμ(u,s)$, contracting $\bar G$ onto $H'$.
    Further we obtain an action of $\Gm$ on $\bar Γ$ that keeps the first $X$ coordinate fixed.
    By construction this action contracts $\bar Γ$ onto a closed subvariety of $\bar Γ \setminus Γ$.
    We will denote this subvariety by $Z₁$ and the contraction morphism $π₁\colon \bar Γ → Z₁$ by $π₁$.

    We can also lift the action to $\schemeclsY[S₁,S₂]\stack X$ where it contracts onto $Z$.
    We will denote the corresponding contraction morphism by $π\colon \schemeclsY[S₁,S₂]\stack X → Z$.
    The morphism $\schemebarq ∘ α_{S₁,S₂}\colon \schemeclsY[S₁,S₂]\stack X → \bar Γ$ is equivariant with respect to these $\Gm$-actions and on its image the action keeps the second $X$-coordinate fixed.

    We note that the D-modules $\schemej_!\sheaf F$ and $α_{S₁,S₂}^! \schemebarq^! \schemej_! \sheaf F$ are monodromic with respect to these $\Gm$-actions.
    Thus the contraction principle Theorem~\ref{thm:pre:contraction_principle} implies that
    \[
        (\schemei_H)^* α_{S₁,S₂}^! \schemebarq^! \schemej_! \sheaf F = π_* α_{S₁,S₂}^! \schemebarq^! \schemej_! \sheaf F.
    \]
    By construction, the square
    \[
        \begin{tikzcd}
            \schemecls[S₁,S₂] \stack X \arrow{r}{π} \arrow{d}{\schemebarq ∘ α_{S₁,S₂}} & Z \arrow{d} \\
            \bar Γ \arrow{r}{π₁} & Z₁
        \end{tikzcd}
    \]
    is Cartesian.
    Let us call the right vertical map $f$.
    Base change yields
    \[
        π_* α_{S₁,S₂}^!\schemebarq^!\schemej_! \sheaf F =
        f^!π_{1,*} \schemej_! \sheaf F.
    \]
    Finally, let $i_{Z₁}\colon Z₁ \hookrightarrow \bar Γ$ be the inclusion.
    Applying the contraction principle again we obtain
    \[
        f^!π_{1,*} \schemej_! \sheaf F =
        f^!i_{Z₁}^* \schemej_! \sheaf F =
        0.
        \qedhere
    \]
\end{proof}

\begin{proof}[Proof of Theorem~\ref{thm:torus:is-good}]
    By Lemma~\ref{lem:stabilizers_cover} the closed subschemes $\schemeclsY[S]\stack X$ for $S ∈ \Stab \stack X$ cover $\schemeclsY\stack X$.
    If $S₁,\, S₂ ∈ \Stab\stack X$, then
    \[
        \schemeclsY[S₁]\stack X ∩ \schemeclsY[S₂]\stack X =
        \schemeclsY[S₁∩S₂,\, S₁S₂].
    \]
    Thus the theorem follows from an iterated Mayer-Vietoris argument using Lemma~\ref{lem:key_for_stablizier}.
\end{proof}


\printbibliography

\end{document}
