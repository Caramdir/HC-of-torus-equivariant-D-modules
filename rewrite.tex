\documentclass{ck-article}

\title{Hochschild cohomology of torus equivariant D-modules: New approach}
\author{Clemens Koppensteiner}

\usepackage{math-ag}
\addbibresource{math.bib}

\newcommand\bc{\textbf{($\mathbf{*}$)}}

\newcommand\cs{\stack{B}}

\begin{document}
\maketitle

\section{Base change along the diagonal}

\begin{Def}
    Let $\stack X$ be any QCA stack.
    Consider the Cartesian diagram
    \[
        \begin{tikzcd}
            \ls \stack X \arrow[d, "p_1"] \arrow[r, "p_2"] & \stack X \arrow[d, "\Delta"] \\
            \stack X \arrow[r, "\Delta"] & \stack X \times \stack X
        \end{tikzcd}
    \]
    We say that $\stack X$ has property \bc\ if the canonical morphism
    \[
        p_{1,!}p_2^! \to \Delta^!\Delta_!
    \]
    of endofunctors of $\catDModHol{\stack X}$ is an equivalence.
\end{Def}

\begin{Lem}
    Property \bc\ holds whenever the diagonal $\Delta\colon \stack X \to \stack X \times \stack X$ is either proper or smooth.
    In particular it holds for schemes and classifying stacks $\cs{G}$.
\end{Lem}

\subsection{Some reduction steps}

\begin{Lem}\label{lem:bc_lc_immersion}
    If $i\colon\stack Y \hookrightarrow \stack X$ is a locally closed immersion and $\stack X$ has property \bc, then $\stack Y$ has property \bc.
\end{Lem}

[Fix the following proof notationally.]

\begin{proof}
    Let us first assume that $i$ is open.
    It suffices to show that $p_{\stack Y,1,!}p_{\stack Y,2}^!i^! \to \Delta_{\stack Y}^!\Delta_{\stack Y,!}i^!$ is an equivalence.
    Now, using the assumption on $\stack X$ and the fact that $i^! = i^*$ we get
    \[
        p_{\stack Y,1,!} p_{\stack Y,2}^!i^! =
        p_{\stack Y,1,!} i^! p_{\stack X,2}^! \cong
        i^! p_{\stack X,1,!} p_{\stack X,2}^! \isoto
        i^! \Delta_{\stack X}^! \Delta_{\stack X,!} =
        \Delta_{\stack Y}^! (i\times i)^! \Delta_{\stack X,!} \cong
        \Delta_{\stack Y}^! \Delta_{\stack X,!} i^!.
    \]
    Now let $i$ be a closed immersion.
    As above, it suffices to show that $i_!p_{\stack Y,1,!}p_{\stack Y,2}^!i^! \to i_!\Delta_{\stack Y}^!\Delta_{\stack Y,!}i^!$ is an equivalence.
    Now, using the assumption on $\stack X$ and the fact that $i_! = i_*$ we get
    \[
        i_! p_{\stack Y,1,!} p_{\stack Y,2}^!i^! =
        p_{\stack X,1,!} i_! p_{\stack Y,2}^!i^! \cong
        p_{\stack X,1,!} p_{\stack X,2}^! i_!i^! \isoto
        \Delta_{\stack X}^! \Delta_{\stack X,!} i_!i^! \cong
        \Delta_{\stack Y}^! (i\times i)_! \Delta_{\stack X,!}i^! =
        i_! \Delta_{\stack Y}^! \Delta_{\stack X,!} i^!.
        \qedhere
    \]

\end{proof}

\begin{Lem}\label{lem:bc_open_cover}
    If $\stack X$ has an open cover by substacks $\stack U_i$ with property $\bc$, then $\stack X$ has property $\bc$.
\end{Lem}

\begin{Lem}\label{lem:bc_product}
    If $\stack{X_1}$ and $\stack{X_2}$ both have property \bc, then so does $\stack{X_1} \times \stack{X_2}$.
\end{Lem}

\begin{proof}
    The category $\catDMod{\stack X_1 \times \stack X_2}$ is generated by elements of the form $\sheaf F_1 \boxtimes \sheaf F_2$ with $\sheaf F_i \in \catDModHol{\stack X_i}$.
    For any schematic morphism $f\colon \stack Y \to \stack Z$ of QCA stacks the functor $f^!$ is continuous on $\catDMod{\stack Z}$ and $f_!$ is preservers colimits in $\catDModHol{\stack Y}$, since it has a right adjoint.
    Thus it is sufficient to check that property \bc\ holds on elements of the form $\sheaf F_1 \boxtimes \sheaf F_2 \in \catDModHol{\stack X_1 \times \stack X_2}$.
    This is now clear because all functors respect $\boxtimes$.
\end{proof}

% Is there a strengthening to fiber products?

\subsection{Contractible stacks}

\begin{Lem}
    Assume that $\stack X$ is contractible and both $\stack X_0$ and $\stack X\setminus \stack X_0$ have property \bc.
    Then so does $\stack X$.
\end{Lem}

\section{Torus quotients}

\begin{Thm}
    Let $X$ be a normal variety with an action of a torus $T$.
    Then $\stack X = X/T$ has property \bc.
\end{Thm}

\begin{proof}[Proof idea]
    By Sumihiro's theorem \cite{Sumihiro:1974:EquivariantCompletions} and Lemma~\ref{lem:bc_open_cover} we can assume that $\stack X$ is affine with a linear action of $T$.
    By Lemma~\ref{lem:bc_lc_immersion} it further suffices to consider the case that $\stack X = \as{m}/T$ for some $n$.
    By Lemma~\ref{lem:bc_product}, we can further decompose $\as{m}$ into $T$-eigenspaces.
    Each of those splits into copies of $\as{1}/T \cong \as{1}/\Gm \times BT'$.
    This can now be done by a direct calculation using the contraction principle.
\end{proof}

\printbibliography

\end{document}
