\documentclass{ck-article}

\title{Hochschild cohomology of torus equivariant D-modules: New approach}
\author{Clemens Koppensteiner}

\usepackage{math-ag}
\addbibresource{math.bib}

\newcommand\bc{\textbf{($\mathbf{*}$)}}
\newcommand\hbc{\textbf{($\mathbf{**}$)}}

\newcommand\cs{\stack{B}}

\begin{document}
\maketitle

\section{Base change along the diagonal}

\begin{Def}
    Let $\stack X$ be any QCA stack.
    Consider the Cartesian diagram
    \[
        \begin{tikzcd}
            \ls \stack X \arrow[d, "p_1"] \arrow[r, "p_2"] & \stack X \arrow[d, "\Delta"] \\
            \stack X \arrow[r, "\Delta"] & \stack X \times \stack X
        \end{tikzcd}
    \]
    We say that $\stack X$ has property \bc\ if the canonical morphism
    \[
        p_{1,!}p_2^! \to \Delta^!\Delta_!
    \]
    of endofunctors of $\catDModHol{\stack X}$ is an equivalence.
\end{Def}

\begin{Lem}\label{lem:bc_smooth_or_proper}
    Property \bc\ holds for $\stack X$ whenever the diagonal $\Delta\colon \stack X \to \stack X \times \stack X$ is either proper or smooth.
    In particular it holds for (separated) schemes and classifying stacks $\cs{G}$.
\end{Lem}

\subsection{Some reduction steps}

Let $i\colon\stack Y \to \stack X$ be an open (or closed) immersion.
We have $\ls Y \cong \stack Y \times_{\stack X \times \stack X} \stack Y$ and it follows that there is a canonical open (resp.~closed) immersion of loop stacks $\ls i\colon \ls \stack Y \to \ls \stack X$.

\begin{Lem}\label{lem:bc_lc_immersion}
    If $i\colon\stack Y \hookrightarrow \stack X$ is a locally closed immersion and $\stack X$ has property \bc, then $\stack Y$ has property \bc.
\end{Lem}

\begin{proof}
    Let us first assume that $i$ is open.
    It suffices to show that $p_{\stack Y,1,!}p_{\stack Y,2}^!i^! \to \Delta_{\stack Y}^!\Delta_{\stack Y,!}i^!$ is an equivalence.
    Now, using the assumption on $\stack X$ and the fact that $i^! = i^*$ we get
    \[
        p_{\stack Y,1,!} p_{\stack Y,2}^!i^! =
        p_{\stack Y,1,!} (\ls i)^! p_{\stack X,2}^! \cong
        i^! p_{\stack X,1,!} p_{\stack X,2}^! \isoto
        i^! \Delta_{\stack X}^! \Delta_{\stack X,!} =
        \Delta_{\stack Y}^! (i\times i)^! \Delta_{\stack X,!} \cong
        \Delta_{\stack Y}^! \Delta_{\stack X,!} i^!.
    \]
    Now let $i$ be a closed immersion.
    As above, it suffices to show that $i_!p_{\stack Y,1,!}p_{\stack Y,2}^!i^! \to i_!\Delta_{\stack Y}^!\Delta_{\stack Y,!}i^!$ is an equivalence.
    Now, using the assumption on $\stack X$ and the fact that $i_! = i_*$ we get
    \[
        i_! p_{\stack Y,1,!} p_{\stack Y,2}^!i^! =
        p_{\stack X,1,!} (\ls i)_! p_{\stack Y,2}^!i^! \cong
        p_{\stack X,1,!} p_{\stack X,2}^! i_!i^! \isoto
        \Delta_{\stack X}^! \Delta_{\stack X,!} i_!i^! \cong
        \Delta_{\stack Y}^! (i\times i)_! \Delta_{\stack X,!}i^! =
        i_! \Delta_{\stack Y}^! \Delta_{\stack X,!} i^!.
        \qedhere
    \]
\end{proof}

\begin{Lem}\label{lem:bc_open_cover}
    If $\stack X$ has an open cover by substacks $\stack U_i$ with property \bc, then $\stack X$ has property \bc.
\end{Lem}

\begin{Lem}\label{lem:bc_product}
    If $\stack{X_1}$ and $\stack{X_2}$ both have property \bc, then so does $\stack{X_1} \times \stack{X_2}$.
\end{Lem}

\begin{proof}
    The category $\catDMod{\stack X_1 \times \stack X_2}$ is generated by elements of the form $\sheaf F_1 \boxtimes \sheaf F_2$ with $\sheaf F_i \in \catDModHol{\stack X_i}$.
    For any schematic morphism $f\colon \stack Y \to \stack Z$ of QCA stacks the functor $f^!$ is continuous on $\catDMod{\stack Z}$ and $f_!$ is preservers colimits in $\catDModHol{\stack Y}$, since it has a right adjoint.
    Thus it is sufficient to check that property \bc\ holds on elements of the form $\sheaf F_1 \boxtimes \sheaf F_2 \in \catDModHol{\stack X_1 \times \stack X_2}$.
    This is now clear because all functors respect $\boxtimes$.
\end{proof}

% Is there a strengthening to fiber products?

\subsection{Contractible stacks}

\begin{Lem}\label{lem:bc_contractive}
    Assume that $\stack X$ has a trivial contractive $\Gm$-action and that both $\stack X_0$ and $\stack X\setminus \stack X_0$ have property \bc.
    Then so does $\stack X$.
\end{Lem}

\begin{proof}
    Let $i\colon \stack X_0 \hookrightarrow \stack X$ and let $j\colon \stack U =\stack X\setminus \stack X_0 \hookrightarrow \stack X$ be its complement.
    Consider the distinguished triangle
    \[
        i_!i^! \Delta^!\Delta_! \to \Delta^!\Delta_! \to j_*j^*\Delta^!\Delta_!
    \]
    on $\catDModHol{\stack X}$.
    It suffices to show that the canonical maps on the outside terms to $i_!i^! p_{1,!}p_2^!$ and $j_*j^* p_{1,!}p_2^!$ are isomorphisms.
    
    Using the fact that $j^* = j^!$, base change and the assumption we see that
    \[
        j_*j^*\Delta^!\Delta_! \cong
        j_*\Delta^!\Delta_!j^* \isoto
        j_*p_{\stack U,1,!}p_{\stack U,2}^! j^* \cong
        j_*j^* p_{1,!}p_2^!
    \]
    is an equivalence.

    Write $\pi\colon \stack X \to \stack X_0$ for the contraction map opposite to $i$.
    Then by the contraction principle,
    \begin{equation}\label{eq:lem:bc_contractive}
        i_!i^! \Delta^!\Delta_! =
        i_!\Delta_{\stack X_0}^!(i\times i)^! \Delta_! \cong
        i_!\Delta_{\stack X_0}^!(\pi\times \pi)_! \Delta_! =
        i_!\Delta_{\stack X_0}^!\Delta_{\stack X_0,!} \pi_! \cong
        i_!\Delta_{\stack X_0}^!\Delta_{\stack X_0,!} i^!.
    \end{equation}
    Write $\ls i \colon \ls\stack X_0 \hookrightarrow \ls\stack X$ for the corresponfing closed inclusion of loop stacks.
    We note that $\ls\stack X$ inherits a contractive $\Gm$ action form $\stack X$.
    Let $\ls \pi\colon \ls\stack X \to \ls\stack X_0$ be the corresponding contraction map.
    Then by assumption and the contraction principle, \eqref{eq:lem:bc_contractive} is equivalent to
    \[
        i_!p_{\stack X_0,1,!}p_{\stack X_0,2}^! i^! =
        i_!p_{\stack X_0,1,!} (\ls i)^! p_{2}^! \cong
        i_!p_{\stack X_0,1,!} (\ls \pi)_! p_{2}^! =
        i_! \pi_! p_{1,!} p_{2}^! =
        i_! i^! p_{1,!} p_{2}^!. \qedhere
    \]
\end{proof}

\section{Compatibility base changes}

Let us write $\stack G_\bullet = \stack G_\bullet(\stack X)$ for the nerve of the groupoid associated to $\ls\stack X \rightrightarrows \stack X$.
Thus
\[
    \stack G_m(\stack X) = \underbrace{\stack X \times_{\stack X \times \stack X} \dotsb \times_{\stack X \times \stack X} \stack X}_{m+1\text{ times}}.
\]
If $i\colon \stack Y \hookrightarrow \stack X$ is a closed (or open) substack, then $\stack G_m(\stack Y)$ is a closed (resp.~open) substack of $\stack G_m(\stack X)$.
Sbsuing notation, we will continue to denote this inclusion by $i$.

\begin{Def}
    Consider the cartesian square
    \[
        \begin{tikzcd}
            \stack G_{n+1} \arrow[r] \arrow[d] & \stack G_n \arrow[d] \\
            \stack G_{m+1} \arrow[r] & \stack G_m
        \end{tikzcd}
    \]
    [describe maps].
    We say that $\stack X$ has property \hbc\ if the canonical morphism
    \[
        ...
    \]
    of functors $\catDModHol{\stack G_n} \to \catDModHol{\stack G_m+1}$ is an equivalence.
\end{Def}

\begin{Lem}
    If the diagonal  morphism $\Delta\colon \stack X \to \stack X \times \stack X$ is proper, then property \hbc\ holds for $\stack X$.
    In particular it holds for (separated) schemes.
\end{Lem}

\begin{proof}
    It suffices to show the base change for face and degeneracy maps, where the vertical arrows turn out to be proper.
    [Fill in the details.]
\end{proof}

\begin{Lem}
    Let $G$ be an affine algebraic group. 
    Then property \hbc\ holds  for the classifying stack $\cs{G}$
\end{Lem}

\begin{Lem}\label{lem:hbc_lc_immersion}
    If $i\colon\stack Y \hookrightarrow \stack X$ is a locally closed immersion and $\stack X$ has property \bc, then $\stack Y$ has property \bc.
\end{Lem}

\begin{proof}
    As for Lemma~\ref{lem:bc_lc_immersion}.
\end{proof}

\begin{Lem}\label{lem:hbc_open_cover}
    If $\stack X$ has an open cover by substacks $\stack U_i$ with property \hbc, then $\stack X$ has property \hbc.
\end{Lem}

\begin{Lem}\label{lem:hbc_product}
    If $\stack{X_1}$ and $\stack{X_2}$ both have property \hbc, then so does $\stack{X_1} \times \stack{X_2}$.
\end{Lem}

\begin{proof}
    Denote by $\stack G_\bullet(\stack X_i)$ the groupoid associated to $\stack X_i \to \stack X_i \times \stack X_i$.
    Then we have a natural decomposition $\stack G_m(\stack X_1 \times \stack X_2) \cong \stack G_m(\stack X_1) \times \stack G_m(\stack X_2)$.
    Thus the method of the proof of Lemma~\ref{lem:bc_product} can also be applied to show the corresponding statement for property \hbc.
\end{proof}

\section{Torus quotients}

\begin{Thm}
    Let $X$ be a normal variety with an action of a torus $T$.
    Then $\stack X = X/T$ has property \bc.
\end{Thm}

\begin{proof}[Proof idea]
    By Sumihiro's theorem \cite{Sumihiro:1974:EquivariantCompletions} and Lemma~\ref{lem:bc_open_cover} we can assume that $\stack X$ is affine with a linear action of $T$.
    By Lemma~\ref{lem:bc_lc_immersion} it further suffices to consider the case that $\stack X = \as{m}/T$ for some $n$.
    By Lemma~\ref{lem:bc_product}, we can further decompose $\as{m}$ into $T$-eigenspaces.
    Each of those splits into copies of $\as{1}/T \cong \as{1}/\Gm \times BT'$.
    
    By Lemma~\ref{lem:bc_smooth_or_proper} it remains to consider the stack $\as{1}/\Gm$, where $\Gm$ acts nontrivially.
    If the $\Gm$ action is contractive this case follows immediatley from Lemma~\ref{lem:bc_contractive} and Lemma~\ref{lem:bc_smooth_or_proper}.
    [Say something about the anti-contractive case.]
\end{proof}

For property \hbc\ one can do a similar reduction to a computation for $\as{1}/\Gm$.
For the special case the contraction principle is again a useful tool.

\printbibliography

\end{document}
